\chapter{Aufgabenstellung}
\label{sec:aufgabenstellung}
%In der Aufgabenstellung wird (in eigenen Worten und ganzen Sätzen) formuliert, was das Ziel des 
%Versuches ist.  
%[Beachten Sie die eigentliche Aufgabenstellung in den Versuchsanleitungen sowie die Hinweise zur Auswertung!] 

Im Versuch 3 "`Abwasserbeschaffenheit II"' werden drei Abwasserproben unbekannter Herkunft über Abwasser-Feststoffuntersuchungen und Abwasser-Summenparameter analysiert. Konkret untersucht werden dafür das Sedimentationsverhalten, das Absetzvolumen (AV), abfiltrierbare Stoffe in Form der Trockensubstanz (TS), organische Trockensubstanz (oTS), der chemische Sauerstoffbedarf (CSB) und der biochemische Sauerstoff über 5 Tage hinweg (BSB$_5$).\\
Ziel der Auswertung, der gesammelten Messdaten, ist eine Einschätzung der Herkunft und Belastung der Abwasserproben, sowie ein Vergleich der jeweiligen Beschaffenheit mit häuslichem Abwasser und den Mindestanforderungen für das Einleiten kommunaler Abwässer in einen Vorfluter der GK 5. \\
Im Anschluss sind Empfehlungen zur Abwasserbehandlung zu geben.