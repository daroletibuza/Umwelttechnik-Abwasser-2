\chapter{Durchführung}
\label{sec:durchfuerung}

Im ersten Versuchsteil werden die drei Proben mittels Sinnesprüfung untersucht. Alle Proben werden hierfür durchgeschüttelt, um homogenisierte Wasserproben zu erhalten und anschließend in Klarglasgefäße, in diesem Fall drei Erlenmeyerkolben, umgefüllt. Vor einem weißen Hintergrund platziert (siehe Abb. \ref{fig:proben}), erfolgt nun die optische Sinnesprüfung mit Einschätzung von Färbung und Trübung nach den Einstufungen in Tab. \ref{tab:einstufungen}. 



Im zweiten Versuchsteil werden die Proben elektrochemisch mittels pH-Wert-Elektrode und 2-in-1 Potentiometer-Temperatursensor analysiert. Die pH-Wert-Elektrode ist dabei auf \SI{25}{\celsius} kalibriert und untersucht werden die drei Abwasserproben einmal mit und einmal ohne vorangegangene Vakuumfiltration (siehe Abb. \ref{fig:proben_filter}). \linebreak Während der elektrochemischen Messungen wird die jeweils zu untersuchende Probe mittels Magnetrührer bei ca. \SI{300}{\rpm} homogen gehalten.


Im dritten und letzten Versuchsabschnitt werden die zuvor gefilterten Proben auf ihre enthaltenen Ionen geprüft. Dafür werden die Proben wiedermals durch schütteln homogenisiert und dann im ersten Zug via Schnellteststreifen analysiert. Es werden Schnellteststreifen der Firma \textsc{Chemsolute$^{\textsuperscript{\textregistered}}$} genutzt, welche die Proben auf Nitrit-, Nitrat- und Phosphat-Ionen testen und mittels abgestufter Farbskala eine grobe Beurteilung über den Gehalt der Ionen in $\left[\si{\milli \gram}\right]$ ermöglichen. Beispielhaft sind in Abbildung \ref{fig:phosphat_test} die Schnelltestpackungen mit den Farbskalen und danebenliegenden Phosphat-Teststreifen zu sehen.\\
