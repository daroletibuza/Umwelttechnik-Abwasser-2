\chapter{Durchführung}
\label{sec:durchfuerung}
Im ersten Versuchsteil werden die drei Proben mittels Abwasser-Feststoffuntersuchungen, sprich über das Sedimentationsverhalten, das Absetzvolumen (AV), die abfiltrierbaren Stoffe in Form der Trockensubstanz (TS) und der organischen Trockensubstanz (oTS), analysiert. In allen Versuchsteilabschnitten werden die Proben zuvor durch Schütteln homogenisiert.\\
Die Bestimmung des Sedimentationsverhaltens erfolgt durch einfüllen der drei Proben in je einen Imhoff-Trichter. Es wird Periodisch die Lage der Trennschicht abgelesen, welche sich zwischen der klaren und der Schwebstoffphase ausbildet.
Die Werte wurden 10 mal minütlich und danach alle 5 Minuten aufgenommen. Der letzte Wert wird nach insgesamt 30 Minuten abgelesen und ist gleichzeitig das Absetzvolumen. \\
Die Trockensubstanz wird durch Abfiltrieren einer bestimmten Probenmenge bestimmt. Vorher muss das verwendete Filterpaper getrocknet und eingewogen werden. Der erhaltene Filterkuchen wird auf dem Filterpapier bei \SI{105}{\degreeCelsius} getrocknet und zuletzt gewogen.\\
Der organische Anteil der Trockensubstanz erfolgt analog der DIN 38409. Der zuvor erhaltene Filterkuchen wird 2 Stunden lang bei \SI{550}{\degreeCelsius} verascht.\\

Im zweiten Versuchsteil werden die Proben über die Abwasser-Summenparameter in Form des chemischen Sauerstoffbedarfs (CSB) und des biochemischen Sauerstoffbedarfs über 5 Tage (BSB$_5$) untersucht.\\

Der chemische Sauerstoffbedarf wird mittels CSB-Testampulle bestimmt. Die Ampullen wurden mit den zu untersuchenden Proben befüllt, geschüttelt und in den vorgewärmten CSB-Reaktor zur Inkubation eingelegt und darin für 2 Stunden belassen. Auf den Deckeln der Proben ist der Inhalt zu beschriften. Es muss eine sinnvolle Verdünnung der Probe gewählt werden. In diesem Falle wurde für die Probe 1 ein hoher CSB erwartet. Darum wurde die Probe 1 ein mal im Verhältnis 1:1 mit Wasser gemischt, und ein mal pur analysiert. Die eigentliche Analyse wurde mit Hilfe des Hach-Messgerätes entsprechend der Betriebsanweisung durchgeführt.\\

Die Vorbereitung der Proben für die Ermittelung des Biochemischen Sauerstoffbedarfes begann mit der Befüllung der Analyseflaschen. In diese wurden je ein Magnetrührstäbchen, \SI{10}{\milli\liter} vorbereitete Nährstofflösung und 2 Dosiereinheiten Nitrifikationshemmen gegeben bevor die eigentliche Waserprobe hinzugegeben wurde. Der eingelegte Dichtkörper wurde in jeder Flasche mit je 2 Plätzchen Natriumhydroxid bestückt. Die Dichtung des Deckels muss mit etwas Schlifffett bestrichen werden bevor er handfest aufgeschraubt wird. Anschließend werden die Analyseflaschen in den Messschrank auf Magnetrührplatten gestellt und an die Innenliegenden Schläuche angeschlossen. Der Name des genutzten manometrischen Messsystems der Firma Nanotec lautet BSB-Win2000. Der BSB wird über einen Zeitraum von 5 Tagen mittels der Differenzdruckmethode bestimmt.

