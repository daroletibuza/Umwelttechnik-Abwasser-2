%Dokumentklasse

%draft als optionohne bilder für bessere performance
%\documentclass[a4paper,12pt,draft]{scrreprt}

%normal mit Bildern
\documentclass[a4paper,12pt]{scrreprt}

\usepackage[left= 3cm,right = 3cm, bottom = 3cm,top = 3cm]{geometry}
%\usepackage[onehalfspacing]{setspace}

% ============= Packages =============
% Dokumentinformationen
\usepackage[
pdftitle={Praktikum - Umwelttechnik},
pdfsubject={},
pdfauthor={Roman-Luca Zank},
pdfkeywords={},	
%Links nicht einrahmen
hidelinks
]{hyperref}

%nur Text zum prüfen des Umfangs

% Standard Packages
%\usepackage[bottom]{footmisc}
\usepackage[utf8]{inputenc}
\usepackage[ngerman]{babel}

\usepackage[T1]{fontenc}
%\usepackage{helvet}

%\renewcommand{\familydefault}{\sfdefault}

\usepackage{graphicx}
\graphicspath{{img/}}
\usepackage{mhchem}
\usepackage{fancyhdr}
\usepackage{lmodern}
\usepackage{color}
\usepackage{placeins}
\usepackage{booktabs}
\usepackage{caption}
\usepackage[list=true]{subcaption}
\usepackage{longtable}
\usepackage{tikz}
\usepackage{pgfplots}
\usepackage{lastpage}
\usetikzlibrary{patterns}

%Einheitenpackage
\usepackage{siunitx}  
\sisetup{	locale = DE, 
	per-mode=fraction,
	inter-unit-product=\ensuremath{\cdot},
	detect-weight = true,
	quotient-mode=fraction
}
%neue Einheiten definieren
\DeclareSIUnit\xyz{xyz}	
\DeclareSIUnit\rpm{rpm}	

%Automatisch cdot statt *
\DeclareMathSymbol{*}{\mathbin}{symbols}{"01}



%\setcounter{lofdepth}{2}      %für subfigures in list of figures

%Tabelle
\usepackage{tabularx}
\usepackage{tabulary}

%%Verzeichnispackages
%\usepackage{natbib}

%\usepackage{hyperref}
%\newcites{

% zusätzliche Schriftzeichen der American Mathematical Society
\usepackage{amsfonts}
\usepackage{amsmath}

%Abkürzungsverzeichnis
\usepackage{acronym}

%kein Abstand bei neuem Kapitel vom Seitenanfang
%\vspace*{2.3\baselineskip} = ORIGINAL
\renewcommand*{\chapterheadstartvskip}{\vspace*{.0\baselineskip}}

%nicht einrücken nach Absatz
\setlength{\parindent}{0pt}


\urlstyle{same}

% ============= Kopf- und Fußzeile =============
\pagestyle{fancy}
%
\lhead{}
\chead{}
\rhead{}%\slshape }%\leftmark}
%%
\lfoot{}
\cfoot{}
\rfoot[{\thepage\ of \pageref*{LastPage}}]{Seite \thepage\ von \pageref*{LastPage}}
%%
\renewcommand{\headrulewidth}{0pt}
\renewcommand{\footrulewidth}{0pt}
\renewcommand{\chapterpagestyle}{fancy}

%Fußnotelinie
%\let\footnoterule

%Fußnote mit Klammer
\renewcommand*{\thefootnote}{(\arabic{footnote})}

%Abb. statt Abbildung
\addto\captionsngerman{%
	\renewcommand{\figurename}{Abb.}%
	\renewcommand{\tablename}{Tab.}%
}

% ============= Package Einstellungen & Sonstiges ============= 
%Besondere Trennungen
%\hyphenation{De-zi-mal-tren-nung}
\usepackage[none]{hyphenat}
\hyphenpenalty=5000
\tolerance=5000
\providecommand\phantomsection{}

\usepackage{mathtools}


% ============= Dokumentbeginn =============

\begin{document}
%Seiten ohne Kopf- und Fußzeile sowie Seitenzahl
\pagestyle{empty}

%\begin{center}
\begin{tabular}{p{\textwidth}}


\begin{center}
\includegraphics[scale=0.75]{logos.jpg}\\
\end{center}


\\

\begin{center}
\LARGE{\textsc{
Protokoll \\
Umwelttechnik\\
}}
\end{center}

\\

%\begin{center}
%\large{Fakultät für Muster und Beispiele \\
%der Hochschule Musterhausen \\}
%\end{center}
%
%\\

\begin{center}
\textbf{\Large{V3 - Abwasserbeschaffenheit II}}
\end{center}

\begin{center}
	\large{Gruppe 1.2 (BCUT3)}
\end{center}


\\
%\begin{center}
%zur Erlangung des akademischen Grades\\
%Bachelor of Engineering
%\end{center}


%\begin{center}
%vorgelegt von
%\end{center}

\begin{center}
\Large{\textbf{Teilnehmer:}} \\ 
\end{center}
\begin{center}
\large{Christoph Hecht \\
	Willy Messerschmidt \\
	Roman-Luca Zank} \\
\end{center}


\\

\begin{center}
\begin{tabular}{lll}
\large{\textbf{Protokollführer:}} & & \large{Roman-Luca Zank} \\
& & \href{mailto:roman-luca.zank@stud.hs-merseburg.de}{{\footnotesize roman-luca.zank@stud.hs-merseburg.de}}\\
&&\\
\large{\textbf{Datum der Versuchsdurchführung:}}&& \large{17.12.2019}\\
&&\\
\large{\textbf{Abgabedatum:}}&& \large{14.01.2020}
\end{tabular}
\end{center}

\\ \\ \\ \\ \\ \\ 
\large{Merseburg den \today}

\end{tabular}
\end{center}


%\include{14_danksagungen}

%\include{15_zusammenfassung}

% Beendet eine Seite und erzwingt auf den nachfolgenden Seiten die Ausgabe aller Gleitobjekte (z.B. Abbildungen), die bislang definiert, aber noch nicht ausgegeben wurden. Dieser Befehl fügt, falls nötig, eine leere Seite ein, sodaß die nächste Seite nach den Gleitobjekten eine ungerade Seitennummer hat. 
\cleardoubleoddpage

% Pagestyle für Titelblatt leer
\pagestyle{empty}

%Seite zählen ab
\setcounter{page}{0}

%Titelblatt
\begin{center}
\begin{tabular}{p{\textwidth}}


\begin{center}
\includegraphics[scale=0.75]{logos.jpg}\\
\end{center}


\\

\begin{center}
\LARGE{\textsc{
Protokoll \\
Umwelttechnik\\
}}
\end{center}

\\

%\begin{center}
%\large{Fakultät für Muster und Beispiele \\
%der Hochschule Musterhausen \\}
%\end{center}
%
%\\

\begin{center}
\textbf{\Large{V3 - Abwasserbeschaffenheit II}}
\end{center}

\begin{center}
	\large{Gruppe 1.2 (BCUT3)}
\end{center}


\\
%\begin{center}
%zur Erlangung des akademischen Grades\\
%Bachelor of Engineering
%\end{center}


%\begin{center}
%vorgelegt von
%\end{center}

\begin{center}
\Large{\textbf{Teilnehmer:}} \\ 
\end{center}
\begin{center}
\large{Christoph Hecht \\
	Willy Messerschmidt \\
	Roman-Luca Zank} \\
\end{center}


\\

\begin{center}
\begin{tabular}{lll}
\large{\textbf{Protokollführer:}} & & \large{Roman-Luca Zank} \\
& & \href{mailto:roman-luca.zank@stud.hs-merseburg.de}{{\footnotesize roman-luca.zank@stud.hs-merseburg.de}}\\
&&\\
\large{\textbf{Datum der Versuchsdurchführung:}}&& \large{17.12.2019}\\
&&\\
\large{\textbf{Abgabedatum:}}&& \large{14.01.2020}
\end{tabular}
\end{center}

\\ \\ \\ \\ \\ \\ 
\large{Merseburg den \today}

\end{tabular}
\end{center}
 %Prokolle
%\include{01_titel2} %Seminar-/Abschlussarbeit

% Pagestyle für Rest des Dokuments
\pagestyle{fancy}

%Inhaltsverzeichnis
\tableofcontents
\thispagestyle{empty}

%Inhalt

%%Verzeichnis aller Bilder
\label{sec:bilder}
\listoffigures
\addcontentsline{toc}{chapter}{Abbildungsverzeichnis}
\thispagestyle{empty}

%Verzeichnis aller Tabellen
\label{sec:tabellen}
\listoftables
\addcontentsline{toc}{chapter}{Tabellenverzeichnis}
\thispagestyle{empty}

%Abkürzungsverzeichnis
\setlength{\columnsep}{20pt}
%\twocolumn
%\addchap{Nomenklatur}
%\label{sec:abkurzung}
%\begin{acronym}
%\acro{kf}[$\text{k}_\text{f}$]{Durchlässigkeitsbeiwert}
%\acro{t}{Durchlaufzeit}
%\acro{tm}[$\text{t}_\text{m}$]{Mittlere Durchlaufzeit}
%\acro{V}{Volumen}
%\acro{h}{Höhe der Wassersäule}
%\acro{Q}{Volumenstrom}
%\acro{l}{Durchströmte Länge}
%\acro{A}{Grundfläche}
%\acro{d}{Durchmesser}
%
%\end{acronym}
%\subsubsection{Aufrufen einer Abkürzung}
%\acs{rT}
%\begin{verbatim}
%\acs{Abkürzung}
%\end{verbatim}

\chapter{Aufgabenstellung}
\label{sec:aufgabenstellung}
%In der Aufgabenstellung wird (in eigenen Worten und ganzen Sätzen) formuliert, was das Ziel des 
%Versuches ist.  
%[Beachten Sie die eigentliche Aufgabenstellung in den Versuchsanleitungen sowie die Hinweise zur Auswertung!] 

Im Versuch 3 "`Abwasserbeschaffenheit II"' werden drei Abwasserproben unbekannter Herkunft ... \\
Ziel der Auswertung, der gesammelten Messdaten, ist eine Einschätzung der Herkunft und Belastung der Abwasserproben, sowie ein Vergleich der jeweiligen Beschaffenheit mit häuslichem Abwasser und den Mindestanforderungen für das Einleiten kommunaler Abwässer in einen Vorfluter der GK 5. \\
Im Anschluss sind Empfehlungen zur Abwasserbehandlung zu geben.

\chapter{Geräte und Chemikalien}
\label{sec:geraete}

\textbf{Geräte:}
\begin{itemize}
	\item Magnetrührer mit Rührfisch
	\item Bechergläser
	\item Erlenmeyerkolben
	\item Filterpapier
\end{itemize}

\vspace*{5mm}

\textbf{Proben/Chemikalien:}
\begin{itemize}
	\item destilliertes Wasser
	\item Abwasserproben 1, 2 \& 3
\end{itemize}
%Start




\chapter{Durchführung}
\label{sec:durchfuerung}
Im ersten Versuchsteil werden die drei Proben mittels Abwasser-Feststoffuntersuchungen, sprich über das Sedimentationsverhalten, das Absetzvolumen (AV), die abfiltrierbaren Stoffe in Form der Trockensubstanz (TS) und der organischen Trockensubstanz (oTS), analysiert. In allen Versuchsteilabschnitten werden die Proben zuvor durch Schütteln homogenisiert.\\
Die Bestimmung des Sedimentationsverhaltens erfolgt durch einfüllen der drei Proben in je einen Imhoff-Trichter. Es wird Periodisch die Lage der Trennschicht abgelesen, welche sich zwischen der klaren und der Schwebstoffphase ausbildet.
Die Werte wurden 10 mal minütlich und danach alle 5 Minuten aufgenommen. Der letzte Wert wird nach insgesamt 30 Minuten abgelesen und ist gleichzeitig das Absetzvolumen. \\
Die Trockensubstanz wird durch Abfiltrieren einer bestimmten Probenmenge bestimmt. Vorher muss das verwendete Filterpaper getrocknet und eingewogen werden. Der erhaltene Filterkuchen wird auf dem Filterpapier bei \SI{105}{\degreeCelsius} getrocknet und zuletzt gewogen.\\
Der organische Anteil der Trockensubstanz erfolgt analog der DIN 38409. Der zuvor erhaltene Filterkuchen wird 2 Stunden lang bei \SI{550}{\degreeCelsius} verascht.\\

Im zweiten Versuchsteil werden die Proben über die Abwasser-Summenparameter in Form des chemischen Sauerstoffbedarfs (CSB) und des biochemischen Sauerstoffbedarfs über 5 Tage (BSB$_5$) untersucht.\\

Der chemische Sauerstoffbedarf wird mittels CSB-Testampulle bestimmt. Die Ampullen wurden mit den zu untersuchenden Proben befüllt, geschüttelt und in den vorgewärmten CSB-Reaktor zur Inkubation eingelegt und darin für 2 Stunden belassen. Auf den Deckeln der Proben ist der Inhalt zu beschriften. Es muss eine sinnvolle Verdünnung der Probe gewählt werden. In diesem Falle wurde für die Probe 1 ein hoher CSB erwartet. Darum wurde die Probe 1 ein mal im Verhältnis 1:1 mit Wasser gemischt, und ein mal pur analysiert. Die eigentliche Analyse wurde mit Hilfe des Hach-Messgerätes entsprechend der Betriebsanweisung durchgeführt.\\

Die Vorbereitung der Proben für die Ermittelung des Biochemischen Sauerstoffbedarfes begann mit der Befüllung der Analyseflaschen. In diese wurden je ein Magnetrührstäbchen, \SI{10}{\milli\liter} vorbereitete Nährstofflösung und 2 Dosiereinheiten Nitrifikationshemmen gegeben bevor die eigentliche Waserprobe hinzugegeben wurde. Der eingelegte Dichtkörper wurde in jeder Flasche mit je 2 Plätzchen Natriumhydroxid bestückt. Die Dichtung des Deckels muss mit etwas Schlifffett bestrichen werden bevor er handfest aufgeschraubt wird. Anschließend werden die Analyseflaschen in den Messschrank auf Magnetrührplatten gestellt und an die Innenliegenden Schläuche angeschlossen. Der Name des genutzten manometrischen Messsystems der Firma Nanotec lautet BSB-Win2000. Der BSB wird über einen Zeitraum von 5 Tagen mittels der Differenzdruckmethode bestimmt.



\chapter{Ergebnisse}
\label{sec:ergebnisse}

Im folgenden Protokollabschnitt werden die Versuchsergebnisse der Versuchsdurchführung präsentiert.
\vspace*{-3.5mm}

\section{Sedimentationsverhalten}
\section{Absetzvolumen}
\section{Trockensubstanz TS}
\section{organische Trockensubstanz oTS}
\section{chemischer Sauerstoffbedarf CSB}
\section{biologischer Sauerstoffbedarf BSB$_5$}

\newpage

\section{Gegenüberstellung der Mindestanforderungen für das Einleiten kommunaler Abwässer in einen Vorfluter der GK 5 mit den Abwasserproben}
Die Referenzwerte der Mindestanforderungen für das Einleiten kommunaler Abwässer in den Vorfluter der Größenklasse 5 sind im Anhang von \cite[S. 29]{Skript} zu finden.
\vspace*{-2.5mm}
\renewcommand{\arraystretch}{1.2}
\begin{table}[h!]
	\centering
	\caption{Tabellarischer Vergleich der Messwerte mit den Mindestanforderungen für das Einleiten kommunaler Abwässer in den Vorfluter der GK 5}
	\label{tab_vgl}
	%\resizebox{10cm}{!}{
	\begin{tabulary}{1.2\textwidth}{l|C|C}
		\hline
		 & \textbf{$CSB$} $\boldsymbol{\left[\si{\milli\gram\per\liter}\right]}$ & \textbf{$BSB_5$} $\boldsymbol{\left[\si{\milli\gram\per\liter}\right]}$\\
		\hline
		\textbf{Grenzwert} & \textbf{75} & \textbf{15}  \\
		\hline
		Probe 1 &  &  \\
		Probe 2 &  &  \\
		Probe 3 &  &  \\
		\hline
	\end{tabulary}
	%}
\end{table}
\FloatBarrier

\begin{figure}[h!]
	\begin{tikzpicture}
	\selectcolormodel{gray}
	\begin{axis}[
	xbar=1pt,% space of 0pt between adjacent bars
	bar width=7,
	width=15cm,
	height=7cm,
	%minor y tick num=4,
	xmax=90,xmin=0,
	x tick label style={/pgf/number format/.cd,%
		scaled x ticks = false,
		set decimal separator={,},
		fixed},
	symbolic y coords={Probe 3,Probe 2,Probe 1,max. für GK 5},
	ytick=data,
	%legend style={at={(0.5,-0.15)},
	%	anchor=north,legend columns=-1},
	xtick={0,10,...,90},
	grid=major,
	xlabel=Gehalt in  \si{\milli\gram\per\liter},
	%enlargelimits=0.15,
	%postaction={pattern=north east lines}
	]
	%CSB
	\addplot[fill=black] coordinates {
		(0,Probe 1) (1.3,Probe 2) (0.6,Probe 3) (75,max. für GK 5)
	};
	%BSB5
	\addplot coordinates {
		(50,Probe 1) (0.77,Probe 2) (2.18,Probe 3) (15,max. für GK 5)
	};

	\legend{$CSB$,$BSB_5$}
	\end{axis}
	\end{tikzpicture}
	\caption{Vergleich mit Mindestanforderungen für das Einleiten kommunaler Abwässer in den Vorfluter der GK 5 für die Abwasserproben 1 bis 3}
	\label{Balkendiagramm}
\end{figure}
\FloatBarrier


\newpage

\section{Gegenüberstellung der durchschnittlichen Beschaffenheit von häuslichem Abwasser mit den Abwasserproben}

Um die Messwerte des Versuches mit häuslichem Abwasser gegenüberzustellen wird die Tabelle Tab. \ref{tab:komm} (siehe \cite[S. 29]{Skript}) genutzt.

\vspace*{0.5cm}
\renewcommand{\arraystretch}{1.2}
\begin{table}[h!]
	\centering
	\caption[Tabellenausschnitt zur durchschnittlichen Beschaffenheit von häuslichem Abwasser]{Tabellenausschnitt zur durchschnittlichen Beschaffenheit von häuslichem Abwasser \cite[S. 29]{Skript}}
	\label{tab:komm}
	%\resizebox{10cm}{!}{
	\begin{tabulary}{1.2\textwidth}{l|C|C|C|C}
	\textbf{Kriterium} 		& \textbf{Maßeinheit} 				&\multicolumn{3}{c}{\textbf{Belastungsgrad}}\\
	\hline
							&									& gering	& mittel & stark\\
	\hline
	Absetzbare Stoffe		&\si{\milli \liter \per \liter} 	& 2			& 6	 	 & 12\\
	Abfiltrierbare Stoffe	&\si{\milli \gram \per \liter} 		& 200		&500	 & 900\\
	$CSB$					&\si{\milli \gram \per \liter} 		& 300		&600	 & 1000\\
	$BSB_5$					&\si{\milli \gram \per \liter}		& 150		&300	 & 500\\
	\end{tabulary}
	%}
\end{table}
\FloatBarrier
\vspace*{1.5cm}

\begin{figure}[h!]
	\begin{tikzpicture}
	\begin{axis}[
	%x tick label style={
	%	/pgf/number format/1000 sep=},
	xtick = data,
	ylabel=Absetzbare Stoffe in \si{\milli \liter \per \liter},
   	enlarge x limits=0.15,
	ybar,
	ymin = 0,
	ymax = 15,
	bar width=10pt,
	width=15cm,
	height=9cm,
	symbolic x coords={0,Probe 1, Probe 2,Probe 3,2,1},
	]
	\addplot[fill=gray!50,draw=black!60] coordinates {(Probe 1,1.7) (Probe 2,7.38) (Probe 3,7.44)
	};
	\addplot[green!50!black,sharp plot,update limits=false, dashed] 
	coordinates {(0,2) (1,2)} 
	node[above] at (axis cs:Probe 2,2) {\hspace*{-5.5cm}geringe Belastung
	};
	\addplot[orange,sharp plot,update limits=false, dashed] 
	coordinates {(0,6) (1,6)} 
	node[above] at (axis cs:Probe 2,6) {\hspace*{-5.5cm}mittelere Belastung
	};
	\addplot[red,sharp plot,update limits=false, dashed] 
	coordinates {(0,12) (1,12)} 
	node[above] at (axis cs:Probe 2,12) {\hspace*{-5.5cm}starke Belastung
	};
	\legend{Absetzbare Stoffe}
	\end{axis}
	\end{tikzpicture}
	\caption{Absetzbare Stoffe der Abwasserproben 1 bis 3}
	\label{dia:absetz}
\end{figure}
\FloatBarrier

\begin{figure}[h!]
	\begin{tikzpicture}
	\begin{axis}[
	%x tick label style={
	%	/pgf/number format/1000 sep=},
	xtick = data,
	ylabel=Abfiltrierbare Stoffe in \si{\milli \gram \per \liter},
	enlarge x limits=0.15,
	ybar,
	ymin = 0,
	ymax = 1200,
	ytick ={0,200,500,900,1200},
	bar width=10pt,
	width=15cm,
	height=6.7cm,
	symbolic x coords={0,Probe 1, Probe 2,Probe 3,2,1},
	]
	\addplot[fill=gray!50,draw=black!60] coordinates {(Probe 1,108.7) (Probe 2,0.77) (Probe 3,2.18)
	};
	\addplot[green!50!black,sharp plot,update limits=false, dashed] 
	coordinates {(0,200) (1,200)} 
	node[above] at (axis cs:Probe 2,200) {geringe Belastung
	};
	\addplot[orange,sharp plot,update limits=false, dashed] 
	coordinates {(0,500) (1,500)} 
	node[above] at (axis cs:Probe 2,500) {mittelere Belastung
	};
	\addplot[red,sharp plot,update limits=false, dashed] 
	coordinates {(0,900) (1,900)} 
	node[above] at (axis cs:Probe 2,900) {starke Belastung
	};
	\legend{Abfiltrierbare Stoffe}
	\end{axis}
	\end{tikzpicture}
	\caption{Abfiltrierbare Stoffe der Abwasserproben 1 bis 3}
	\label{dia:abfilt}
\end{figure}
\FloatBarrier

\begin{figure}[h!]
	\begin{tikzpicture}
	\begin{axis}[
	%x tick label style={
	%	/pgf/number format/1000 sep=},
	xtick = data,
	ylabel=$CSB$ in \si{\milli \gram \per \liter},
	enlarge x limits=0.15,
	ybar,
	ymin = 0,
	ymax = 1400,
	ytick ={0,300,600,1000,1400},
	bar width=10pt,
	width=15cm,
	height=6.7cm,
	symbolic x coords={0,Probe 1, Probe 2,Probe 3,2,1},
	]
	\addplot[fill=gray!50,draw=black!60] coordinates {(Probe 1,0) (Probe 2,3.9) (Probe 3,8.8)
	};
	\addplot[green!50!black,sharp plot,update limits=false, dashed] 
	coordinates {(0,300) (1,300)} 
	node[above] at (axis cs:Probe 2,300) {geringe Belastung
	};
	\addplot[orange,sharp plot,update limits=false, dashed] 
	coordinates {(0,600) (1,600)} 
	node[above] at (axis cs:Probe 2,600) {mittelere Belastung
	};
	\addplot[red,sharp plot,update limits=false, dashed] 
	coordinates {(0,1000) (1,1000)} 
	node[above] at (axis cs:Probe 2,1000) {starke Belastung
	};
	\legend{CSB}
	\end{axis}
	\end{tikzpicture}
	\caption{Chemischer Sauerstoffbedarf (CSB) der Abwasserproben 1 bis 3}
	\label{dia:csb}
\end{figure}
\FloatBarrier

\begin{figure}[h!]
	\begin{tikzpicture}
	\begin{axis}[
	%x tick label style={
	%	/pgf/number format/1000 sep=},
	xtick = data,
	ylabel=$BSB_5$ in \si{\milli \gram \per \liter},
	enlarge x limits=0.15,
	ybar,
	ymin = 0,
	ymax = 800,
	ytick ={0,150,300,500,700},
	bar width=10pt,
	width=15cm,
	height=6.7cm,
	symbolic x coords={0,Probe 1, Probe 2,Probe 3,2,1},
	]
	\addplot[fill=gray!50,draw=black!60] coordinates {(Probe 1,0) (Probe 2,3.9) (Probe 3,8.8)
	};
	\addplot[green!50!black,sharp plot,update limits=false, dashed] 
	coordinates {(0,150) (1,150)} 
	node[above] at (axis cs:Probe 2,150) {geringe Belastung
	};
	\addplot[orange,sharp plot,update limits=false, dashed] 
	coordinates {(0,300) (1,300)} 
	node[above] at (axis cs:Probe 2,300) {mittelere Belastung
	};
	\addplot[red,sharp plot,update limits=false, dashed] 
	coordinates {(0,500) (1,500)} 
	node[above] at (axis cs:Probe 2,500) {starke Belastung
	};
	\legend{BSB$_5$}
	\end{axis}
	\end{tikzpicture}
	\caption{Biochemischer Sauerstoffbedarf über 5 Tage (BSB$_5$) der Abwasserproben 1 bis 3}
	\label{dia:bsb}
\end{figure}
\FloatBarrier



\chapter{Diskussion}
\label{sec:diskussion}
In diesem Abschnitt des Protokolls werden nun die Ergebnisse des Abschnittes \ref{sec:ergebnisse} diskutiert und ausgewertet.\\\\

%Die erhaltenen Analysenergebnisse sind tabellarisch und grafisch darzustellen.  

%%In Ergebnissen meiner meinung nach ausreichend erfolgt

%Die Werte sind den Mindestanforderungen für das Einleiten kommunaler Abwässer (GK 5; siehe 
%Anlage des Praktikumsheftes) gegenüberzustellen und zu diskutieren. 
Alle drei untersuchten Abwasserproben erreichen nicht die Mindestanforderungen für das Einleiten in den Vorfluter für die GK5. Dabei sticht die Probe 1 besonders heraus. Sie überschreitet den Grenzwert von \SI{75}{\milli\gram
	\per\liter} $CSB$ um das 7,6-Fache und den Grenzwert von \SI{15}{\milli\gram\per\liter} $BSB_5$ sogar um das knapp 17-Fache. Die Proben 2 und 3 können die Grenzwerte ebenfalls bei weitem nicht erfüllen, was in Abb. \ref{Balkendiagramm} dargestellt ist. 


%Ordnen Sie die Proben hinsichtlich ihres Belastungsgrades ein!
  
  Zur Einordnung der Proben hinsichtlich ihrer Belastung werden im Folgenden die in den Tabellen \ref{tab:absetzvol}, \ref{tab:filter}, \ref{tab:csb} und \ref{tab:bsb} aufgeführten Ergebnisse mit den Referenzwerten in Tabelle \ref{tab:komm} verglichen.
  
Die Proben 1 und 2 sind den $CSB$- und $BSB_5$-Werten nach mittelstark mit organischem Material belastet, wärend die Probe 3 als gering belastet einzustufen ist.
Ein weiteres Merkmal für den Belastungsgrad stellt der Anteil absetzbarer und abfiltrierbarer Stoffe dar. Die Probe mit \SI{90}{\milli\gram
	\per\liter} sehr gering belastet, wohingegen die Probe 2 mit \SI{210}{\milli\gram
	\per\liter} schon gering und die Probe 1 mit \SI{1319}{\milli\gram
	\per\liter} sehr stark belastet ist.
Die Betrachtung hinsichtlich absetzbarer Stoffe ergibt für die Probe 1 eine sehr starke Belastung mit einem Volumenanteil von \SI{210}{\milli\liter\per\liter} nach nur 30 Minuten. Von einer starken Belastung wird ab einem Wert von \SI{12}{\milli\liter\per\liter} ausgegangen. Die Proben 2 und 3 fallen bei \SI{2}{\milli\liter\per\liter} bis \SI{3}{\milli\liter\per\liter} beide unter die geringe Belastungsstufe.\\

%Streichen Sie signifikante Unterschiede zwischen den einzelnen Proben heraus und schließen Sie auf 
%ihre Herkunft! 
!!!!!!!!!!!!!!!!!!!!!!!!!!!!!!!!!!!!!!!!!!!!!!!\\


%Was bedeuten CSB und BSB5 ? Erklären Sie die Merkmale und den Stellenwert beider Analysen.
Der $CSB$-Wert gibt an wie viel Sauerstoff von einem  starken chemischen Oxidationsmittel, wie etwa Kaliumdichromat, zur Oxidation aller im Wasser enthaltenen oxidierbaren Stoffe verbraucht wird. Neben den biologischen und organischen Stoffen werden zum Teil aber auch anorganische Verbindungen oxidiert. Darum liegt der $CSB$ in der Regel höher als der $BSB_5$.\\
Der $BSB_5$ gibt an wie viel Sauerstoff bei der biologischen Oxidation im Wasser befindlicher organischer Stoffe verbraucht wird. Nicht alle organischen Inhaltsstoffe können innerhalb der gewährten Zeit oxidiert werden. Außerdem werden circa 50\% der organischen Stoffe für das Wachstum der Mikroorganismen benötigt und ist somit nicht oxidiert.\cite[S.64]{rosenwinkelAnaerobtechnikAbwasserSchlamm2015} \\
Die Angabe des $CSB$ und $BSB_5$ ermöglichen eine Einordnung der Abwässer hinsichtlich ihres Gehaltes an Biomasse (organischen Stoffen). Ins besondere bei der Abschätzung des Gefahrenpotentials des Abwassers für aquatische Ökosysteme sind der $CSB$ als auch der $BSB_5$ unerlässich.

 
%Warum beträgt die Dauer des BSB-Versuches 5 Tage? 
Die Dauer des BSB-Versuches beträgt 5 Tage, weil die verwendeten Mikroorganismen einige Zeit brauchen um sich entsprechend zu vermehren und die angebotene Biomasse zu verstoffwechseln. 5 Tage sind außerdem eine realistische Verweilzeit für Abwässer in einer herkömmlichen Kläranlage.

%Welches CSB/BSB5-Verhältnis besitzt biologisch gut abbaubares Abwasser, welches 
%„Problemwässer“? 

Bei kommunalen Abwässern ist ein Verhältnis von $CSB$ zu $BSB_5$ von etwa 2:1 häufig anzutreffen. 
Ist das Verhältnis kleiner als 2 kann eine gute Abbaubarkeit erwartet werden. Bei Werten größer denn 2 ist keine einfache Schlussfolgerung möglich. Das Wasser muss dann auf andere Arten untersucht werden. "`Problemabwässer"' entstammen zumeist industriellen Quellen. Schadstoffe welche in der natürlichen Umgebung sehr  selten auftreten bedürfen zumeist spezieller Destruenten zum biologischen Abbau. Extrem große Verhältnisse von $CSB$ zu $BSB_5$ lassen darauf schließen, dass der $BSB$ sehr gering ausgeprägt ist. Geringe mikrobielle Aktivität hat demzufolge eine schlechte Abbaubarkeit zur Folge.\cite[S.64]{rosenwinkelAnaerobtechnikAbwasserSchlamm2015}\\


%Errechnen Sie das bestehende CSB/BSB5-Verhältnis und beurteilen Sie die 
%biologische Abbaubarkeit der einzelnen Proben. 

Das $\frac{CSB}{BSB_5}$-Verhältnis der Probe 1 beträgt 2,29. Die Probe 1 gilt damit als mäßig gut abbaubar.
Die Probe 2 ist mit einem $\frac{CSB}{BSB_5}$-Verhältnis von 1,93 gut abbaubar. Eine schlechte Abbaubarkeit bescheinigt der Probe 3 ihr $\frac{CSB}{BSB_5}$-Verhältnis von 4,42.
\\



%Geben Sie Empfehlungen zu einer anforderungsgerechten Behandlung der analysierten Proben. 

Die Proben 1 und 2 können einer kommunalen Kläranlage zugeführt werden. Der enorme Gehalt absetzbarer und abfiltrierbarer Stoffe könnte damit behoben werden, das Abwasser vor dem Einleiten in die Kanalisation durch ein Absetzbecken zu leiten. Andernfalls muss auf eine gute Spülung der Kanalrohre geachtet werden. Die Probe 3 sollte unter Umständen genauer Analysiert werden um den Grund ihrer schlechten biologischen Abbaubarkeit heraus zu finden. Ein zu großer Anteil des Abwassers 3 könnte den Klärprozess, durch Herabsetzung der biologischen Abbaubarkeit des gesamten Abwassers, verlangsamen und behindern. 

%Berechnen Sie für die Abwasserschlammproben den Schlammvolumenindex (SVI)! 
%Nach MOHLMANN gibt dieser das Volumen von 1 g Trockensubstanz des Belebtschlammes nach einer 
%Absetzzeit von 30 min an und stellt sich als Verhältnis spez. Volumen des abgesetzten Schlammes in 
%ml/1000ml und spez. Masse der Trockensubstanz in g/1000ml dar. 

%Welche Aussage können Sie mit dem SVI für Ihre Proben machen?

\textcolor{red}{Probe 2 war die selbe wie in Abwasser I. Probe 1 und Probe 3 sind nicht die selben!}




\chapter{Fehlerbetrachtung}
\label{sec:fehler}
In diesem Abschnitt erfolgt die Fehlerbetrachtung des Versuches, welche Einfluss auf die Messergebnisse haben können.\\

Mess- und Ablesefehler können bei der Bestimmung des Absetzvolumens im \textsc{Imhoff}trichter, bei der automatisierten Messung des $CSB$, bei der Bestimmung des $BSB_5$ nach der Differenzdruckmethode, bei allen durchgeführten Wägevorgängen und Volumenmessungen mit Messzylindern und \textsc{Eppendorf}pipetten aufgetreten sein. Auch die Zeitmessung während des Absetzvorgangs ist fehlerbehaftet, da das Ablesen einige Zeit in Anspruch nahm. Darum handelt es sich bei allen Zeitangaben nur um Näherungswerte mit Abweichungen von bis zu 30 Sekunden.\\
Die Belastungsgrade, welchen die Proben zugeordnet wurden, sind nicht klar definiert. Es handelt sich stets um Bereiche mit fließenden Übergängen. So verhält es sich auch mit den Referenzwerten des $\frac{CSB}{BSB_5}$-Verhältnisses. Je nach Quelle variieren diese geringfügig, jedoch ist eine allgemeine Charakterisierung möglich. \\
Die Versuchsdurchführung weist an einigen Stellen ein beachtenswertes Fehlerpotential auf. Die Trocknung des Filterpapiers auf der Trocknungswaage ist recht genau, da die Wage den Gewichtsverlust kontrolliert und erst bei Stagnation der Masse die Trockenmasse anzeigt. Die Vorbereitung der CSB-Ampullen erforderte die Herstellung einer verdünnten Probenlösung. Hierbei kamen nur Messzylinder zum Einsatz. Dem entsprechend kann das angegebene Volumen um wenige Milliliter variieren. Die eigentliche Befüllung der Ampullen mit \textsc{Eppendorf}pipetten dürfte jedoch keinen signifikanten Fehler verursacht haben. \\ 
Beim Verglühen der Filterkuchen wurden die Filterpapiere mitverglüht. Um die so zusätzlich eingetragene Masse an Asche bilanzieren zu können, wurde ein Blatt Filterpapier der selben Charge parallel mit verglüht. Ein Mittelwert aus einer größeren Anzahl verglühter Filterpapierblättchen würde einen belastbareren Wert ergeben, da er die natürliche Streuung der Aschegehalte ausgleicht. Das Verglühen der Proben könnte zudem unvollständig gewesen sein. \\
Die Bestimmung des $BSB_5$ ist stark vom Wohlbefinden der Mikroorganismen abhängig. Eine Unterdosierung von Nähstoffen in der Nährlösung, zu hohe Temperaturschwankungen oder das versehentliche Einbringen von Natriumhydroxid könnte die Lebensumstände der Mikroorganismen in einen suboptimalen Bereich verschieben und damit deren Aktivität hemmen. Zur besseren Vergleichbarkeit der $BSB_5$-Werte, hätte die Bakterienpopulation der Proben zu Beginn der Messung aneinander angeglichen werden müssen. Eine niedrige Anzahl von Destruenten im Abwasser verlängert die Abbaudauer dadurch, dass sich die Population erst entwickeln muss. Im ungünstigsten Fall reicht dann die Zeit von fünf Tagen nicht zum weitest möglichen biologischen Abbau aus. Undichtigkeiten im System würden, beispielsweise bei unzureichender Schlifffettnutzung, die Differenzdruckmessung kompromittieren. 


%Praktikumsskript, Modul ………, Versuch …….., Prof. Musterprof. 
%DIN 12345, Jahr der Veröffentlichung 
%Link der Internetseite, Zugriffsdatum 
%Buchtitel, Autor, Verlag, Veröffentlichungsjahr 


%Literaturverzeichnis Bücher
\bibliography{Literatur}
\bibliographystyle{unsrtdin}
\addcontentsline{toc}{chapter}{Literaturverzeichnis}







%Anhang
%\addcontentsline{toc}{chapter}{Anhang}

%\include{09_erklaerung}

%\chapter{Bausteine}

\section{Beispiel für Tabelle}

%Tabelle START
\vspace*{-2.5mm}
\renewcommand{\arraystretch}{1.2}
\begin{table}[h!]
	\centering
	\caption{Abmessungen der Probekörper vor dem Zugversuch}
	\label{tab:tabelle1}
	%\resizebox{10cm}{!}{
	\begin{tabulary}{\textwidth}{C|CCC}
		\hline
		\textbf{Probe}  &\textbf{Breite [mm]}&\textbf{Dicke [mm]}&\textbf{Anf.-länge[mm]} \\ 
		\hline
		Kupfer (gewalzt) & 12,5 &3,00&50\\
		Kupfer (geglüht) & 12,5&3,00&50\\
		PA6 & 10,0&4,00&50\\
		PP (EPR-30\% Kautschuk) & 9,9 &3,95&50\\
		\hline
	\end{tabulary}
	%}
\end{table}

\FloatBarrier
\vspace*{-2.5mm}
%Tabelle ENDE

\section*{Tabelle mit Itemize}
%Tabelle START
\vspace*{-2.5mm}
\renewcommand{\arraystretch}{1.2}
\begin{table}[h!]
	\centering
	\caption*{Vor- und Nachteile der Geothermie}
	\label{tab:tabelle1}
	\begin{tabulary}{\textwidth}{C|C}
		\hline
		\textbf{Vorteile}  &\textbf{Nachteile} \\ 
		\hline
		&\\
		\begin{minipage}[t]{0.4\textwidth}
			\begin{itemize}
				\item Strom, Wärme und Kälte wird erzeugt
				\item keine saisonalen und tageszeitlichen Schwankungen
				\item 	quasi-regenerativ
				\item 	nachfrage-gerechte Energiebereitstellung
				\item Erzeugungspotenzial sehr hoch 
				\item 	grundsätzlich standortunabhängig
			\end{itemize}
		\end{minipage} & 
		\begin{minipage}[t]{0.4\textwidth}
			\begin{itemize}
				\item hohe Anschaffungskosten
				\item abhängig von geologischen Gegebenheiten
				\item geringer Stromwirkungsgrad (thermodynamisch bedingt)
				\item keine Marktdurchdringung in DE
				\item erfahrene Bauunternehmen notwendig
				\item gute Vorerkundung und Überwachung notwendig
			\end{itemize}
		\end{minipage}\\
	\end{tabulary}
\end{table}
\FloatBarrier
\vspace*{-2.5mm}
%Tabelle ENDE

\newpage

\section*{Beispiel für Skalierbare Tabelle}
%TAbelle Start
\vspace*{-2.5mm}
\renewcommand{\arraystretch}{1.2}
\begin{table}[h!]
	\centering
	\caption*{}
	\resizebox{0.5\textwidth}{!}{
		\begin{tabulary}{\textwidth}{C|C|C|C}
			\textbf{Name} & \textbf{Anwendung}&\textbf{Gleichung}&\textbf{Stoffkonstante} \\ 
			\hline  
			KICK& $x_{80_\omega}>\SI{50}{\milli\meter}$ &$e_{KICK}=c_K*log(\frac{x_{80_\omega}}{x_{80_\alpha}})$&$c_K=1,15*\frac{c_B}{\sqrt{0,05\si{\meter}}} \left[\si{\raiseto{2}\meter\per\raiseto{2}\second}\right]$\\
			BOND&$\SI{50}{\micro\meter}<x_{80_\omega}<\SI{50}{\milli\meter}$&$e_{BOND}=c_B*\left(\frac{1}{\sqrt{x_{80_\omega}}}-\frac{1}{\sqrt{x_{80_\alpha}}}\right)$& $c_B$: tabelliert $\left[\si{\raiseto{2,5}\meter\per\raiseto{2}\second}\right]$\\ 
			RITTER& $x_{80_\omega}>\SI{50}{\micro\meter}$&$e_{RITT}=c_R*\left(\frac{1}{x_{80_\omega}}-\frac{1}{x_{80_\alpha}}\right)$&$c_R= 0,5*c_B*\sqrt{\SI{5e-5}{\meter}}$ \\  
	\end{tabulary}}
\end{table}
\FloatBarrier
%Ende TAbelle

\section{Befehlszeilen in Text einfügen}
Befehlzeilen aus der "Latex-Sprache"\ lassen sich nicht ohne weiteres im Text darstellen. Das System erkennt diese als solche und gibt warnhinweise aus. In der Regel werden die Befehle dann auch falsch dargestellt. Umgehen lässt sich diese Problematik mit der verbatim-Umgebung. In ihren Grenzen werden Eingaben 1:1 dargestellt wie eingegeben. Die Funktionsweise soll nachfolgend durch ein Beispiel verdeutlicht werden.
\begin{verbatim*}

\begin{verbatim}
\usepackage{Beispieltrolllllollllolll} 
\end{verbatim}

\end{verbatim*}

Sollte besonderer Wert auf die Kenntlichmachung der Leerzeichen gelegt werden kann auch mit \texttt{verbatim*} gearbeitet werden.

\section{Seitenübergreifende, lange Tabellen}

Tabellen welche Messwerte in einem solchen Umfang enthalten, dass sie nicht auf einer einzelnen A4-Seite Platz finden, können mit dem Paket \begin{verbatim}
\usepackage{longtable} 
\end{verbatim}
in ein Dokument eingepflegt werden wie folgendes Beispiel belegt.:
\begin{longtable}[c]{lllll}
	\caption{Dehnungstabelle}\\
	\label{alles}
	$Zeit [HH:MM:SS]$ & $\Delta l_{PE} [mm]$ & $\varepsilon_{PE}$ & $\Delta l_{Pb} [mm]$ & $\varepsilon_{Pb}$ \\
	\hline
	\endfirsthead
	%\caption{}\\
	$Zeit [HH:MM:SS]$ & $\Delta l_{PE} [mm]$ & $\varepsilon_{PE}$ & $\Delta l_{Pb} [mm]$ & $\varepsilon_{Pb}$ \\ 
	\hline
	\endhead
	\multicolumn{5}{r}{Fortsetzung auf n{\"a}chster Seite}\\
	\endfoot
	\hline
	\multicolumn{5}{r}{} \\
	\endlastfoot
	% Ab hier kommt der Inhalt der Tabelle
	00:00:00 & 3,00 & 0,11 & 3,30 & 0,12\\
	00:00:10 & 3,80 & 0,14 & 3,42 & 0,13\\
	00:00:20 & 4,15 & 0,16 & 3,53 & 0,13\\
	00:00:30 & 4,43 & 0,17 & 3,68 & 0,14\\
	00:00:40 & 4,61 & 0,17 & 3,71 & 0,14\\
	00:00:50 & 4,80 & 0,18 & 3,81 & 0,14\\
	00:01:00 & 4,95 & 0,19 & 4,12 & 0,15\\
	00:20:00 & 8,03 & 0,30 &  & \\
\end{longtable} 

\section{Diagramme}
Diagramme lassen sich unter anderem mit dem Paket 
\begin{verbatim}
\usepackage{pgfplots} 
\end{verbatim}
implementieren.
Um das Dokument einigermaßen klein zu halten empfehle ich folgendes weiters Paket zu nutzen.
\begin{verbatim}
\usepackage{csvsimple} 
\end{verbatim}
Es erleichter das importieren von Datensätzen aus .csv Dateien. Die .csv Datei muss folgenden Anforderungen genügen:- Dezimaltrenner Punkt
\begin{itemize}
	\item Dezimaltrenner Punkt
	\item Jede Zeile umgebrochen
	\item Koordinaten durch komma getrennt
	\item Spalten beschriften z.B x,y oder a,b
\end{itemize}

Sollte man eine Tabellenkalkulationsdatei in eine csv Datei konvertiert haben, hat dieses meist die Falsche Form. Durch die Funktion suchen und ersetzen (bsp emacs) können aber sehr schnell die nötigen Korrekturen erfolgen. 

\begin{figure}[h]
	\begin{center}
		\begin{tikzpicture}
		\begin{axis}[
		width=12cm,
		height=6cm,
		xlabel=Zeit in Sekunden,
		ylabel=Dehnung]
		
		\begin{scope}[brown]
		\draw[brown] ({axis cs:10,0}|-{rel axis cs:0,1}) -- ({axis cs:10,0}|-{rel axis cs:0,0});
		\draw[brown] ({axis cs:120,0}|-{rel axis cs:0,1}) -- ({axis cs:120,0}|-{rel axis cs:0,0});
		\end{scope} 
		
		\addplot table [x=a, y=b, col sep=comma] {data/KriechkurvePb.csv};
		\end{axis}
		\end{tikzpicture}
		\caption{Kriechkurve Blei}
		\label{kkb}
	\end{center}
\end{figure} 
\FloatBarrier                     

\section*{Beispiel für Berechnungen}
Die Berechnung der wahren Spannung $\sigma_{W}$ bei Höchstkraft erfolgt unter der Annahme, dass der Prüfkörperquerschnitt noch 60\% des Ausgangsquerschnitts beträgt. Die Berechnung erfolgt ab Gleichung \ref{ber1}.\\ 

%Berechnung der Fläche A
\textbf{Fläche} $\boldsymbol{A}$ \textbf{:}
%Start
\begin{flalign}
A 	&= \frac{\pi}{4}*d^2\\
&=\frac{\pi}{4}*(\SI{80}{\milli \meter})^2
\end{flalign}
%Ende

\section{Beispiel für ein Bild}

%Start
\begin{figure}[h!]
	\centering
	\includegraphics[width=0.60\textwidth]{img/skizzepruef3}
	\caption{Skizze Prüfkörperbemaßung}
	\label{skizzepruef}
\end{figure}
\FloatBarrier
%Ende

\newpage

\section{Beispiel für zwei Bilder}
\label{sec:versuchsaufbau}

%Start
\begin{figure}[h!]
	\centering
	\begin{subfigure}{.5\textwidth}
		\centering
		\includegraphics[width=0.75\textwidth]{Aufbau2}
		\caption{Skizze zum Versuchsaufbau}
		\label{fig:sub1}
	\end{subfigure}%
	\begin{subfigure}{.5\textwidth}
		\centering
		\includegraphics[width=0.6\textwidth]{img/Aufbau1}
		\caption{realer Versuchsaufbau}
		\label{fig:sub2}
	\end{subfigure}
	\caption{Versuchsaufbau als Skizze und in Realität}
	\label{fig:aufbau} 
\end{figure}
\FloatBarrier
%Ende

\newpage


\section{Beispiel für vier Bilder}
%Start
\begin{figure}[h!]
	\centering
	\begin{subfigure}{.5\textwidth}
		\centering
		\includegraphics[width=0.75\textwidth]{img/Kupfer_gewalzt}
		\caption{Kupfer (gewalzt)}
		\label{fig:sub3}
	\end{subfigure}%
	\begin{subfigure}{.5\textwidth}
		\centering
		\includegraphics[width=0.75\textwidth]{img/Kupfer_weich}
		\caption{Kupfer (geglüht)}
		\label{fig:sub4}
	\end{subfigure}
	\begin{subfigure}{.5\textwidth}
		\centering
		\includegraphics[width=0.75\textwidth]{img/PA6}
		\caption{PA6}
		\label{fig:sub5}
	\end{subfigure}%
	\begin{subfigure}{.5\textwidth}
		\centering
		\includegraphics[width=0.75\textwidth]{img/PP}
		\caption{PP (ERP-30\% Kautschuk)}
		\label{fig:sub6}
	\end{subfigure}
	
	\caption{Bruchstellennahaufnahmen der Probekörper}
	\label{fig:bruchstellen} 
\end{figure}
\FloatBarrier
%Ende

\section{Beispiel Einheiten}

\begin{align*}
\textbf{$\SI{12,0/12}{\kg\meter\per\second \raiseto{5} \per \xyz}*\SI{13}{\per\raiseto{-2}\meter}=\SI{256}{}$}
\end{align*}

\begin{align}
\SI{12,0/12}{\meter\per\joule}*\SI{13}{\gram}=\SI{256}{\coulomb}
\end{align}

\newpage

\section{Beispiel für Mini-Formelsammlung}
\begin{flalign}
\label{gl1}
\text{\textbf{Dehnung (Def.)} } \boldsymbol{\varepsilon} \text{ \textbf{:}} && \hspace*{-1em}  \varepsilon=\frac{\Delta l}{l_0} &&
\end{flalign}

\begin{flalign}
\label{gl2}
\text{\textbf{norminelle Spannung} } \boldsymbol{\sigma} \text{\textbf{:}} && \hspace*{-3em} \sigma=\frac{F}{A_0} &&
\end{flalign}

\begin{flalign}
\label{gl3}
\text{\textbf{Sekantenmodul (Kunststoffe) }} \boldsymbol{E_S} \text{ \textbf{:}} && E_S=\frac{\sigma_2-\sigma_1}{\varepsilon_2-\varepsilon_1}=\frac{F_2-F_1}{0,002*A_0} &&
\end{flalign}

\begin{flalign}
\label{gl4}
\text{\textbf{E-Modul (Metalle) }} \boldsymbol{E_M} \text{ \textbf{:}} && \hspace*{5em} E_M=\frac{\sigma}{\varepsilon}=\frac{\sigma_2-\sigma_1}{\varepsilon_2-\varepsilon_1}=\frac{R_{p_{0.2\%}}}{0,2\%} &&
\end{flalign}

\begin{flalign}
\label{gl5}
\text{\textbf{Bruchdehnung} } \boldsymbol{A} \text{\textbf{:}} && \hspace*{6em} A= \frac{l_{u}-l_{0}}{l_{0}}*100\% &&
\end{flalign}

\begin{flalign}
\label{gl6}
\text{\textbf{Ausgangsquerschnitt } } \boldsymbol{S_0} \text{\textbf{:}} && \hspace*{3em} S_{0}= Breite*Dicke &&
\end{flalign}
\begin{flalign}
\label{gl7}
\text{\textbf{wahre Spannung} } \boldsymbol{\sigma_{W}} \text{\textbf{:}} &&\hspace*{1em} \sigma_{W}=\frac{F_{max}}{S_{End}}&&
\end{flalign}
\begin{flalign}
\label{gl8}
\text{\textbf{Brucheinschnürung } } \boldsymbol{Z} \text{\textbf{:}} && \hspace*{5em} Z=\frac{S_0-S_u}{S_{o}}*100\% &&
\end{flalign}

\newpage

\section*{Fußnoten}
%Start
\begin{figure}[h!]
	\centering
	\includegraphics[width=0.85\textwidth]{tabdia/kfwerte}
	\caption*{$\text{k}_\text{f}$-Werte der Proben 1 bis 11 \protect\footnotemark[1]}
	\label{}
\end{figure}
\FloatBarrier
%Ende

\footnotetext[1]{bezogen auf $V=\SI{50}{\milli \liter}$ und $h=\SI{10}{\milli \meter}$}



\end{document}
