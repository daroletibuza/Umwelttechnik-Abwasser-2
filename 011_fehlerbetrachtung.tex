\chapter{Fehlerbetrachtung}
\label{sec:fehler}
In diesem Abschnitt erfolgt die Fehlerbetrachtung des Versuches, welche Einfluss auf die Messergebnisse haben können.\\

Mess- und Ablesefehler können bei der Bestimmung des Absetzvolumens im Imhofftrichter, bei der automatisierten Messung des CSB, bei der Bestimmung des BSB nach der Differenzdruckmethode, bei allen durchgeführten Wägevorgängen und Volumenmessungen mit Messzylindern und Eppendorfpipetten aufgetreten sein. Auch die Zeitmessung während des Absetzvorgangs ist Fehlerbehaftet, da das Ablesen einige Zeit in Anspruch nahm. Darum handelt es sich bei allen Zeitangaben nur um Näherungswerte mit Abweichungen von bis zu 30 Sekunden.\\
Die Belastungsgrade, welchen die Proben zugeordnet wurden, sind nicht klar definiert. Es handelt sich stets um interpretationsbedürftige Bereiche. So verhält es sich auch mit den Referenzwerten des $\frac{CSB}{BSB_5}$-Verhältnisses. Je nach Quelle variieren diese geringfügig. \\
Die Versuchsdurchführung weist an einigen Stellen ein beachtenswertes Fehlerpotential auf. Die Trocknung des Filterpapiers auf der Trocknungswage ist recht genau, da die Wage den Gewichtsverlust kontrolliert und erst bei Stagnation der Masse die Trockenmasse anzeigt. Die Vorbereitung der CSB-Ampullen erforderte die Herstellung einer verdünnten Probenlösung. Hierbei kamen nur Messzylinder zum Einsatz. Dem entsprechend kann das angegebene Volumen um wenige Milliliter variieren. Die eigentliche Befüllung der Ampullen mit Eppendorfpipetten dürfte keinen signifikanten Fehler verursacht haben. \\ 
Beim Verglühen der Filterkuchen wurden die Filterpapiere mitverglüht. Um die so zusätzlich eingetragene Masse an Asche herausrechnen zu können, wurde ein Blatt Filterpapier der selben Charge parallel mit verglüht. Ein Mittelwert aus einer größeren Anzahl verglühter Filterpapierblättchen würde einen belastbareren Wert ergeben, da er die natürliche Streuung der Aschegehalte ausgleicht. Das Verglühen der Proben könnte unvollständig gewesen sein. \\
Die Bestimmung des $BSB_5$ ist stark vom Wohlbefinden der Mikroorganismen abhängig. Eine Fehldosierung von Nähstoffen in der Nährlösung, zu hohe Temperaturschwankungen oder das versehentliche Einbringen von Natriumhydroxid könnte die Lebensumstände der Mikroorganismen in einen suboptimalen Bereich verschieben und damit deren Aktivität hemmen. Zur besseren Vergleichbarkeit der $BSB_5$-Werte, hätte die Bakterienpopulation der Proben zu Beginn der Messung aneinander angeglichen werden müssen. Eine niedrige Anzahl von Destruenten im Abwasser verlängert die Abbaudauer dadurch, dass sich die Population erst entwickeln muss. Im ungünstigsten Fall reicht dann die Zeit von fünf Tagen nicht zum weitest möglichen biologischen Abbau aus. Undichtigkeiten im System würden die Differenzdruckmessung kompromittieren. 
