\chapter{Diskussion}
\label{sec:diskussion}
In diesem Abschnitt des Protokolls werden nun die Ergebnisse des Abschnittes \ref{sec:ergebnisse} diskutiert und ausgewertet.\\\\

%Die erhaltenen Analysenergebnisse sind tabellarisch und grafisch darzustellen.  

%Die Werte sind den Mindestanforderungen für das Einleiten kommunaler Abwässer (GK 5; siehe 
%Anlage des Praktikumsheftes) gegenüberzustellen und zu diskutieren. 
%Ordnen Sie die Proben hinsichtlich ihres Belastungsgrades ein!
  
%Streichen Sie signifikante Unterschiede zwischen den einzelnen Proben heraus und schließen Sie auf 
%ihre Herkunft! 

%Was bedeuten CSB und BSB5 ? Erklären Sie die Merkmale und den Stellenwert beider Analysen.
 
%Warum beträgt die Dauer des BSB-Versuches 5 Tage? 

%Welches CSB/BSB5-Verhältnis besitzt biologisch gut abbaubares Abwasser, welches 
%„Problemwässer“? 

%Errechnen Sie das bestehende CSB/BSB5-Verhältnis und beurteilen Sie die 
%biologische Abbaubarkeit der einzelnen Proben. 

%Geben Sie Empfehlungen zu einer anforderungsgerechten Behandlung der analysierten Proben. 

%Berechnen Sie für die Abwasserschlammproben den Schlammvolumenindex (SVI)! 
%Nach MOHLMANN gibt dieser das Volumen von 1 g Trockensubstanz des Belebtschlammes nach einer 
%Absetzzeit von 30 min an und stellt sich als Verhältnis spez. Volumen des abgesetzten Schlammes in 
%ml/1000ml und spez. Masse der Trockensubstanz in g/1000ml dar. 

%Welche Aussage können Sie mit dem SVI für Ihre Proben machen?
 troll


