\chapter{Diskussion}
\label{sec:diskussion}
In diesem Abschnitt des Protokolls werden nun die Ergebnisse des Abschnittes \ref{sec:ergebnisse} diskutiert und ausgewertet.\\\\

%Die erhaltenen Analysenergebnisse sind tabellarisch und grafisch darzustellen.  

%%In Ergebnissen meiner meinung nach ausreichend erfolgt

%Die Werte sind den Mindestanforderungen für das Einleiten kommunaler Abwässer (GK 5; siehe 
%Anlage des Praktikumsheftes) gegenüberzustellen und zu diskutieren. 
Alle drei untersuchten Abwasserproben erreichen nicht die Mindestanforderungen für das Einleiten in den Vorfluter für die GK5. Dabei sticht die Probe 1 besonders heraus. Sie überschreitet den Grenzwert von \SI{75}{\milli\gram
	\per\liter} $CSB$ um das 7,6-Fache und den Grenzwert von \SI{15}{\milli\gram\per\liter} $BSB_5$ sogar um das knapp 17-Fache. Die Proben 2 und 3 können die Grenzwerte ebenfalls bei weitem nicht erfüllen, was in Abb. \ref{Balkendiagramm} dargestellt ist. 


%Ordnen Sie die Proben hinsichtlich ihres Belastungsgrades ein!
  
  Zur Einordnung der Proben hinsichtlich ihrer Belastung werden im Folgenden die in den Tabellen \ref{tab:absetzvol}, \ref{tab:filter}, \ref{tab:csb} und \ref{tab:bsb} aufgeführten Ergebnisse mit den Referenzwerten in Tabelle \ref{tab:komm} verglichen.
  
Die Proben 1 und 2 sind den $CSB$- und $BSB_5$-Werten nach mittelstark mit organischem Material belastet, wärend die Probe 3 als gering belastet einzustufen ist.
Ein weiteres Merkmal für den Belastungsgrad stellt der Anteil absetzbarer und abfiltrierbarer Stoffe dar. Die Probe mit \SI{90}{\milli\gram
	\per\liter} sehr gering belastet, wohingegen die Probe 2 mit \SI{210}{\milli\gram
	\per\liter} schon gering und die Probe 1 mit \SI{1319}{\milli\gram
	\per\liter} sehr stark belastet ist.
Die Betrachtung hinsichtlich absetzbarer Stoffe ergibt für die Probe 1 eine sehr starke Belastung mit einem Volumenanteil von \SI{210}{\milli\liter\per\liter} nach nur 30 Minuten. Von einer starken Belastung wird ab einem Wert von \SI{12}{\milli\liter\per\liter} ausgegangen. Die Proben 2 und 3 fallen bei \SI{2}{\milli\liter\per\liter} bis \SI{3}{\milli\liter\per\liter} beide unter die geringe Belastungsstufe.\\

%Streichen Sie signifikante Unterschiede zwischen den einzelnen Proben heraus und schließen Sie auf 
%ihre Herkunft! 
!!!!!!!!!!!!!!!!!!!!!!!!!!!!!!!!!!!!!!!!!!!!!!!\\


%Was bedeuten CSB und BSB5 ? Erklären Sie die Merkmale und den Stellenwert beider Analysen.
Der $CSB$-Wert gibt an wie viel Sauerstoff von einem  starken chemischen Oxidationsmittel, wie etwa Kaliumdichromat, zur Oxidation aller im Wasser enthaltenen oxidierbaren Stoffe verbraucht wird. Neben den biologischen und organischen Stoffen werden zum Teil aber auch anorganische Verbindungen oxidiert. Darum liegt der $CSB$ in der Regel höher als der $BSB_5$.\\
Der $BSB_5$ gibt an wie viel Sauerstoff bei der biologischen Oxidation im Wasser befindlicher organischer Stoffe verbraucht wird. Nicht alle organischen Inhaltsstoffe können innerhalb der gewährten Zeit oxidiert werden. Außerdem werden circa 50\% der organischen Stoffe für das Wachstum der Mikroorganismen benötigt und ist somit nicht oxidiert.\cite[S.64]{rosenwinkelAnaerobtechnikAbwasserSchlamm2015} \\
Die Angabe des $CSB$ und $BSB_5$ ermöglichen eine Einordnung der Abwässer hinsichtlich ihres Gehaltes an Biomasse (organischen Stoffen). Ins besondere bei der Abschätzung des Gefahrenpotentials des Abwassers für aquatische Ökosysteme sind der $CSB$ als auch der $BSB_5$ unerlässich.

 
%Warum beträgt die Dauer des BSB-Versuches 5 Tage? 
Die Dauer des BSB-Versuches beträgt 5 Tage, weil die verwendeten Mikroorganismen einige Zeit brauchen um sich entsprechend zu vermehren und die angebotene Biomasse zu verstoffwechseln. 5 Tage sind außerdem eine realistische Verweilzeit für Abwässer in einer herkömmlichen Kläranlage.

%Welches CSB/BSB5-Verhältnis besitzt biologisch gut abbaubares Abwasser, welches 
%„Problemwässer“? 

Bei kommunalen Abwässern ist ein Verhältnis von $CSB$ zu $BSB_5$ von etwa 2:1 häufig anzutreffen. 
Ist das Verhältnis kleiner als 2 kann eine gute Abbaubarkeit erwartet werden. Bei Werten größer denn 2 ist keine einfache Schlussfolgerung möglich. Das Wasser muss dann auf andere Arten untersucht werden. "`Problemabwässer"' entstammen zumeist industriellen Quellen. Schadstoffe welche in der natürlichen Umgebung sehr  selten auftreten bedürfen zumeist spezieller Destruenten zum biologischen Abbau. Extrem große Verhältnisse von $CSB$ zu $BSB_5$ lassen darauf schließen, dass der $BSB$ sehr gering ausgeprägt ist. Geringe mikrobielle Aktivität hat demzufolge eine schlechte Abbaubarkeit zur Folge.\cite[S.64]{rosenwinkelAnaerobtechnikAbwasserSchlamm2015}\\


%Errechnen Sie das bestehende CSB/BSB5-Verhältnis und beurteilen Sie die 
%biologische Abbaubarkeit der einzelnen Proben. 

Das $\frac{CSB}{BSB_5}$-Verhältnis der Probe 1 beträgt 2,29. Die Probe 1 gilt damit als mäßig gut abbaubar.
Die Probe 2 ist mit einem $\frac{CSB}{BSB_5}$-Verhältnis von 1,93 gut abbaubar. Eine schlechte Abbaubarkeit bescheinigt der Probe 3 ihr $\frac{CSB}{BSB_5}$-Verhältnis von 4,42.
\\



%Geben Sie Empfehlungen zu einer anforderungsgerechten Behandlung der analysierten Proben. 

Die Proben 1 und 2 können einer kommunalen Kläranlage zugeführt werden. Der enorme Gehalt absetzbarer und abfiltrierbarer Stoffe könnte damit behoben werden, das Abwasser vor dem Einleiten in die Kanalisation durch ein Absetzbecken zu leiten. Andernfalls muss auf eine gute Spülung der Kanalrohre geachtet werden. Die Probe 3 sollte unter Umständen genauer Analysiert werden um den Grund ihrer schlechten biologischen Abbaubarkeit heraus zu finden. Ein zu großer Anteil des Abwassers 3 könnte den Klärprozess, durch Herabsetzung der biologischen Abbaubarkeit des gesamten Abwassers, verlangsamen und behindern. 

%Berechnen Sie für die Abwasserschlammproben den Schlammvolumenindex (SVI)! 
%Nach MOHLMANN gibt dieser das Volumen von 1 g Trockensubstanz des Belebtschlammes nach einer 
%Absetzzeit von 30 min an und stellt sich als Verhältnis spez. Volumen des abgesetzten Schlammes in 
%ml/1000ml und spez. Masse der Trockensubstanz in g/1000ml dar. 

%Welche Aussage können Sie mit dem SVI für Ihre Proben machen?

\textcolor{red}{Probe 2 war die selbe wie in Abwasser I. Probe 1 und Probe 3 sind nicht die selben!}


